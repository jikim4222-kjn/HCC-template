% !TEX program = xelatex
% NOTE: Unicode 文字のため、XeLaTeX(または LuaLaTeX)でコンパイルしてください。
\documentclass[11pt]{article}
\usepackage{fontspec}
\usepackage{xeCJK}
\setmainfont{Latin Modern Roman}
\IfFontExistsTF{Noto Serif CJK JP}{\setCJKmainfont{Noto Serif CJK JP}}{%
  \IfFontExistsTF{Source Han Serif JP}{\setCJKmainfont{Source Han Serif JP}}{%
    \IfFontExistsTF{HaranoAjiMincho}{\setCJKmainfont{HaranoAjiMincho}}{%
      \setCJKmainfont{IPAMincho}% last-resort fallback
    }%
  }%
}%
\usepackage[margin=1in]{geometry}
\usepackage{amsmath}
\usepackage{mathtools}
\usepackage{amssymb}
\usepackage{amsthm}
\usepackage{graphicx}
\usepackage{tikz-cd}
\usepackage{multicol}

\newtheorem{theorem}{Theorem}[section]
\newtheorem{lemma}[theorem]{Lemma}
\newtheorem{proposition}[theorem]{Proposition}
\newtheorem{corollary}[theorem]{Corollary}
\theoremstyle{definition}
\newtheorem{definition}{Definition}[section]

\setlength{\parindent}{0pt}
\setlength{\parskip}{1\baselineskip}

\begin{document}

\title{Hyper Comparison Category (HCC)\\
\large A Categorical Framework for Comparison, Height, and Coarse-Graining}
\author{Jiyoong Kim}
\date{}
\maketitle

\begin{abstract}
We introduce a categorical framework for organizing comparison problems
across multiple observational or coarse--grained levels.
A hyper comparison category consists of projection functors,
an idempotent saturation operator, and invertible $2$--cells
measuring the discrepancy between saturation and observation.
These data give rise to quantitative invariants such as defects,
local supports, and height--type functionals.

Under mild locality and stability assumptions,
defects admit local decompositions and satisfy propagation properties
along towers of projections.
Consequently, inequalities of the form
\[
  \mathrm{defect} \le (1+\varepsilon)\,\mathrm{height} + C
\]
follow from purely structural considerations once suitable external
estimates are provided.

The formalism cleanly separates structural mechanisms from
problem--dependent inputs, and yields a reusable template for reducing
global comparison questions to local bounds together with their
propagation across observational scales.
\end{abstract}

\section{Introduction}

Many mathematical and scientific problems require comparing objects that
do not naturally inhabit a common space, or whose observable features
depend on choices of interpretation, resolution, or coarse--graining.
Examples range from relating models living in different categories, to
connecting structures defined at distinct observational scales, to
reconciling local data across incompatible viewpoints. In such
situations, direct comparison is often ill--posed: the objects of
interest may not admit a canonical identification, and the act of
observation itself may distort or suppress relevant information
\cite{FongSpivak2019,BaezStay2011}.

This paper develops a categorical framework for organizing such
comparison problems. The guiding principle is that comparison should not
be performed on the original objects, but on their \emph{shadows} under
suitable projection functors. Prior to projection, an idempotent
\emph{saturation} operator enforces a chosen interpretation or
coarse--graining; the discrepancy between saturation and observation is
recorded by invertible $2$--cells. Together, these data form what we call
a \emph{hyper comparison category} (HCC).

The HCC formalism isolates three structural mechanisms that recur across
diverse contexts:

\begin{itemize}
  \item a saturation step that fixes an interpretation or abstraction;
  \item a projection step that extracts observable information;
  \item a gauge step measuring the noncommutativity between the two.
\end{itemize}

From these ingredients one obtains quantitative invariants such as
\emph{defects}, \emph{supports}, and \emph{height--type} functionals.
Under mild locality and stability assumptions, defects admit local
decompositions and satisfy monotonicity along towers of projections.
This yields a propagation principle: once suitable local estimates
(\emph{external inputs}) are available at a single observational level,
global inequalities follow automatically across the entire tower.

A key feature of the framework is the clean separation between
\emph{structural} aspects---encoded categorically and valid in complete
generality---and \emph{problem--dependent} analytic or arithmetic
estimates. The former are universal and reusable; the latter enter only
through an external input at a chosen base level. This separation yields
a flexible template for reducing global comparison questions to local
bounds together with their propagation across scales.

The remainder of the paper develops the theory systematically.
Section~2 introduces hyper comparison categories; Section~3 proves the
main comparison theorem; Sections~4 and~5 analyze saturation fixed points
and horizon phenomena; Sections~6 and~7 treat external inputs and their
stability; Section~8 summarizes the resulting protocol; and Section~9
illustrates applications in several settings.

\section{Hyper Comparison Categories}

This section introduces the structures underlying a \emph{hyper
comparison category} (HCC). The purpose is to formalize the interaction
between a normalization step (\emph{saturation}), an observation step
(\emph{projection}), and the discrepancies that appear when these
operations fail to commute. Throughout, $\mathcal C$ denotes a base
category whose objects represent the systems or models to be compared.

\subsection{Basic data}

Fix a partially ordered set $(K,\preceq)$ of observational levels.
An HCC consists of:

\begin{itemize}
  \item a category $\mathcal C$;
  \item for each $k\in K$, an \emph{observable category} $\mathcal O_k$;
  \item projection functors $\Pi_k : \mathcal C \to \mathcal O_k$;
  \item for each $k\preceq k'$, a \emph{coarse--graining functor}
        $\sigma_{k\to k'} : \mathcal O_k \to \mathcal O_{k'}$,
        subject to
        \[
          \Pi_{k'} = \sigma_{k\to k'} \circ \Pi_k.
        \]
\end{itemize}

We additionally require the tower compatibilities
$\sigma_{k\to k} = \mathrm{Id}_{\mathcal O_k}$ and
$\sigma_{k'\to k''}\circ\sigma_{k\to k'} = \sigma_{k\to k''}$ whenever
$k\preceq k'\preceq k''$. The order on $K$ is interpreted as a hierarchy
of resolutions: larger indices correspond to coarser views.

\subsection{Saturation}

A central role is played by an idempotent endofunctor
\[
  S : \mathcal C \to \mathcal C
\]
together with a unit $\eta : \mathrm{Id}_{\mathcal C} \Rightarrow S$.
We refer to $S$ as \emph{saturation}. Intuitively, $S$ enforces a chosen
interpretation, abstraction, or coarse--graining before comparison.

\begin{definition}
A \emph{saturation operator} is an idempotent monad $(S,\eta,\mu)$ on
$\mathcal C$, i.e.\ the multiplication $\mu : S^2 \Rightarrow S$ is an
isomorphism \cite{MacLane1998,Street1972}. An object $X$ is
\emph{saturated} (a \emph{fixed point}) if $\eta_X$ is an isomorphism.
\end{definition}

Saturation provides a canonical normalization step; later, saturated
objects will serve as baselines for ``defect--zero'' comparisons.

\subsection{Observation}

For each $k\in K$, the functor $\Pi_k$ extracts the observable shadow of
an object. Comparisons are performed at the level of $\Pi_k(X)$ rather
than $X$ itself, and the tower relation
$\Pi_{k'} = \sigma_{k\to k'}\circ\Pi_k$ ensures that coarse--graining is
compatible with observation.

\subsection{Gauge and defect}

The interaction between saturation and observation is encoded by
invertible $2$--cells \cite{Leinster2004,Lack2010}. For each morphism
$f : X \to Y$ in $\mathcal C$
and each $k\in K$, we are given an invertible $2$--cell in $\mathcal O_k$
\[
  \Phi_f^{(k)} :
    \Pi_k(Sf)\circ \Pi_k(\eta_X)
    \;\Rightarrow\;
    \Pi_k(\eta_Y)\circ \Pi_k(f).
\]
We call $\Phi_f^{(k)}$ the \emph{gauge} of $f$ at level $k$.

To quantify the size of this discrepancy, assume that each $\mathcal O_k$
comes equipped with a function assigning to every invertible $2$--cell
$\alpha$ a number $|\alpha|_k\in\mathbb R_{\ge 0}$, invariant under
isomorphism of $2$--cells. The \emph{defect} of $f$ at level $k$ is
\[
  C_k(f) := |\Phi_f^{(k)}|_k.
\]

\subsection{Local decomposition}

In many applications, discrepancies decompose into contributions from
independent ``places'' (primes, features, channels, \ldots). Fix a set
$\mathcal P$ and, for each $(k,p)$, a localization functor
\[
  \mathrm{Loc}_{k,p} : \mathcal O_k \to \mathcal O_{k,p}
\]
\cite{Kelly1982}. Assume the size functional decomposes as
\[
  |\alpha|_k = \sum_{p\in\mathcal P} |\mathrm{Loc}_{k,p}(\alpha)|_{k,p}.
\]
Then
\[
  C_k(f) = \sum_{p\in\mathcal P} C_{k,p}(f),
  \qquad
  C_{k,p}(f) := |\mathrm{Loc}_{k,p}(\Phi_f^{(k)})|_{k,p}.
\]
The set $\{p\in\mathcal P : C_{k,p}(f)>0\}$ is the \emph{support} of $f$
at level $k$.

\subsection{Tower monotonicity}

Finally, we impose a compatibility between coarse--graining and defect.

\begin{definition}
An HCC satisfies \emph{tower monotonicity} if for all $k\preceq k'$ and
all invertible $2$--cells $\alpha$ in $\mathcal O_k$,
\[
  |\sigma_{k\to k'}(\alpha)|_{k'} \le |\alpha|_k.
\]
Equivalently, $C_{k'}(f)\le C_k(f)$ for all morphisms $f$.
\end{definition}

Tower monotonicity expresses that coarser observations cannot increase
the apparent discrepancy between saturation and observation.

\medskip

The data and axioms above constitute a hyper comparison category. The
next section shows that these structures force a canonical three--step
factorization of comparison, yielding the main comparison theorem.

\section{The Main Comparison Theorem}

This section records the main structural consequence of the HCC axioms:
comparison at any observational level decomposes canonically into three
steps---saturation, observation, and gauge. In particular, the
quantitative invariants of the theory (defects, supports, and related
height--type functionals) are all induced by this decomposition.

\subsection{The three--step comparison diagram}

Let $f : X \to Y$ be a morphism in $\mathcal C$, and fix an observational
level $k\in K$. The unit $\eta_X : X \to SX$ provides the canonical
saturation map, and applying $S$ to $f$ gives $Sf : SX \to SY$. After
observing at level $k$, we obtain
\[
  \Pi_k(X)
    \xrightarrow{\Pi_k(\eta_X)}
  \Pi_k(SX)
    \xrightarrow{\Pi_k(Sf)}
  \Pi_k(SY)
    \xleftarrow{\Pi_k(\eta_Y)}
  \Pi_k(Y).
\]
The gauge axiom (Axiom~A2) supplies an invertible $2$--cell
\[
  \Phi_f^{(k)} :
    \Pi_k(Sf)\circ \Pi_k(\eta_X)
    \;\Rightarrow\;
    \Pi_k(\eta_Y)\circ \Pi_k(f),
\]
which measures the failure of saturation and observation to commute
along $f$.

\subsection{Statement of the theorem}

\begin{theorem}[Main Comparison Theorem]
\label{thm:MCT}
Let $(\mathcal C,\mathcal O_k,\Pi_k,S,\eta,\Phi_f^{(k)})$ be a hyper
comparison category. For every morphism $f : X \to Y$ and every
$k\in K$:
\begin{enumerate}
  \item[\textnormal{(1)}] \emph{(Three--step factorization)}
  Comparison at level $k$ factors uniquely through saturation and
  observation via the gauge $2$--cell:
  \[
    \Pi_k(Sf)\circ \Pi_k(\eta_X)
    \xRightarrow{\;\Phi_f^{(k)}\;}
    \Pi_k(\eta_Y)\circ \Pi_k(f).
  \]

  \item[\textnormal{(2)}] \emph{(Defect)}
  The defect of $f$ at level $k$,
  \[
    C_k(f) := |\Phi_f^{(k)}|_k,
  \]
  is a well--defined nonnegative quantity, invariant under isomorphism of
  $2$--cells.

  \item[\textnormal{(3)}] \emph{(Local decomposition)}
  Defects decompose over places:
  \[
    C_k(f) = \sum_{p\in\mathcal P} C_{k,p}(f),
    \qquad
    C_{k,p}(f) := |\mathrm{Loc}_{k,p}(\Phi_f^{(k)})|_{k,p}.
  \]

  \item[\textnormal{(4)}] \emph{(Subadditivity)}
  For any composable morphisms $X\xrightarrow{f}Y\xrightarrow{g}Z$,
  \[
    C_k(g\circ f) \le C_k(f) + C_k(g).
  \]

  \item[\textnormal{(5)}] \emph{(Tower monotonicity)}
  If $k\preceq k'$, then
  \[
    C_{k'}(f) \le C_k(f)
    \qquad\text{and}\qquad
    C_{k',p}(f) \le C_{k,p}(f)
    \quad\text{for all } p\in\mathcal P.
  \]
\end{enumerate}
\end{theorem}

\begin{proof}
Item~(1) is exactly Axiom~A2. Item~(2) is the definition of $C_k(f)$ via
Axiom~A3. Item~(3) follows from the local decomposition axiom (Axiom~A5).
Item~(4) is Axiom~A4 applied to the gauge $2$--cells. Item~(5) is tower
monotonicity (Axiom~A6).
\end{proof}

\subsection{Interpretation}

The theorem shows that every comparison in an HCC reduces to the same
canonical three--step process:
\begin{enumerate}
  \item \emph{saturation} (fix an interpretation),
  \item \emph{projection} (extract the observable shadow),
  \item \emph{gauge} (measure the residual discrepancy).
\end{enumerate}
All quantitative invariants arise from the gauge data, and the axioms
ensure that these invariants behave coherently under composition,
localization, and coarse--graining. This rigidity is the mechanism
behind the propagation and convergence results developed later.

\section{BH Normal Form}

Saturation plays a dual role in a hyper comparison category: it both
normalizes objects prior to comparison and singles out a distinguished
class of objects on which defects vanish. These objects form the
\emph{baseline} of the theory; comparisons involving them carry no
residual discrepancy. This section makes these ideas precise and
records the universal property of saturation.

\subsection{Saturated objects and baseline comparison}

An object $X$ is \emph{saturated} if the unit $\eta_X : X \to SX$ is an
isomorphism. Let
\[
  \mathcal C^{\mathrm{BH}} := \{\, X\in\mathcal C : \eta_X \text{ is an isomorphism}\,\}
\]
be the full subcategory of saturated (``BH'') objects.

\begin{lemma}
If $X\in\mathcal C^{\mathrm{BH}}$, then for every $k\in K$,
\[
  C_k(\mathrm{id}_X)=0.
\]
Moreover, if $f:Z\to X$ and $X$ is saturated, then $C_k(f)=0$ for all
$k\in K$.
\end{lemma}

\begin{proof}
If $\eta_X$ is an isomorphism, then the comparison diagram for
$\mathrm{id}_X$ collapses to an identity $2$--cell. Its size is $0$ by
Axiom~A3. The general case is analogous.
\end{proof}

In this sense, saturated objects are \emph{defect--free targets}.

\subsection{Universal property of saturation}

The saturation operator is not merely a normalization procedure: it is a
reflection of $\mathcal C$ onto the baseline subcategory
$\mathcal C^{\mathrm{BH}}$.

\begin{theorem}[BH Normal Form]
\label{thm:BH-normal-form}
For every object $X\in\mathcal C$, the unit $\eta_X : X \to SX$ exhibits
$SX$ as the universal morphism from $X$ to a saturated object. In other
words, for any $B\in\mathcal C^{\mathrm{BH}}$ and any morphism $u : X \to
B$, there exists a unique morphism $\bar u : SX \to B$ such that
\[
  u = \bar u \circ \eta_X.
\]
Equivalently, the inclusion $J : \mathcal C^{\mathrm{BH}} \hookrightarrow
\mathcal C$ admits $S$ as a left adjoint with unit $\eta$.
\end{theorem}

\begin{proof}
This is the standard universal property of an idempotent monad; see
Axiom~A1.
\end{proof}

In particular, every object admits a canonical ``baseline
approximation'' $SX$, and comparisons involving $X$ factor through this
baseline. The comparison diagram of Section~\ref{thm:MCT} always takes
the form
\[
  X \xrightarrow{\eta_X} SX \xrightarrow{Sf} SY \xleftarrow{\eta_Y} Y,
\]
with all ambiguity concentrated in the gauge.

\subsection{Defect collapse on the baseline}

\begin{proposition}
\label{prop:defect-collapse}
If $X$ is saturated, then for every $f : Z \to X$ and every $k\in K$,
\[
  C_k(f) = 0.
\]
More generally, for any morphism $f : X \to Y$,
\[
  C_k(f) = C_k(Sf).
\]
\end{proposition}

\begin{proof}
If $X$ is saturated, then $\eta_X$ is an isomorphism, so the gauge
$\Phi_f^{(k)}$ is isomorphic to an identity $2$--cell; hence $C_k(f)=0$
by Axiom~A3. The second statement follows from functoriality of $S$ and
Axiom~A2.
\end{proof}

Thus saturation behaves as an absorbing state for defects: once an
object is in $\mathcal C^{\mathrm{BH}}$, further saturation does not
change its comparison behavior.

\subsection{Interpretation}

The BH normal form provides a conceptual baseline for comparison:
\begin{itemize}
  \item $SX$ is the canonical interpretation of $X$;
  \item comparisons factor through $SX$ and $SY$;
  \item saturated objects form the defect--free core of the theory;
  \item later results (Sections~5--7) can be viewed as flows toward this
        baseline.
\end{itemize}

\section{Support and Horizon}

Defects in a hyper comparison category decompose canonically over
``places.'' This section develops the resulting geometric intuition: the
set of places that contribute nontrivially to a comparison forms a
support, and its behavior under coarse--graining leads to a natural
\emph{horizon} phenomenon. As the observational level becomes coarser,
visible support can only shrink, and defects decrease monotonically.
These facts provide the structural backbone for the propagation and
convergence results of later sections.

\subsection{Support}

For a morphism $f : X \to Y$ and a level $k\in K$, recall the local
decomposition
\[
  C_k(f) = \sum_{p\in\mathcal P} C_{k,p}(f),
  \qquad
  C_{k,p}(f) := |\mathrm{Loc}_{k,p}(\Phi_f^{(k)})|_{k,p}.
\]
The \emph{support} of $f$ at level $k$ is
\[
  \mathrm{supp}_k(f)
  := \{\, p\in\mathcal P : C_{k,p}(f) > 0 \,\}.
\]
Intuitively, $\mathrm{supp}_k(f)$ records the loci at which the
comparison between $X$ and $Y$ fails to be defect--free (e.g. primes,
features, or independent information channels).

\subsection{Monotonicity of support}

Tower monotonicity (Axiom~A6) implies that coarse--graining does not
increase defect; the same statement holds placewise.

\begin{proposition}[Support monotonicity]
\label{prop:support-monotone}
If $k\preceq k'$, then for all $p\in\mathcal P$,
\[
  C_{k',p}(f) \le C_{k,p}(f),
\]
and consequently
\[
  \mathrm{supp}_{k'}(f) \subseteq \mathrm{supp}_k(f).
\]
\end{proposition}

\begin{proof}
Apply Axiom~A6 to the localized invertible $2$--cells
$\mathrm{Loc}_{k,p}(\Phi_f^{(k)})$.
\end{proof}

Thus the visible support can only shrink as observations become coarser;
categorically, this expresses the principle that coarse--graining
suppresses fine distinctions.

\subsection{Horizon events}

The inclusion in Proposition~\ref{prop:support-monotone} may be strict.

\begin{definition}
A \emph{horizon event} for $f$ between levels $k\preceq k'$ is the strict
containment
\[
  \mathrm{supp}_{k'}(f) \subsetneq \mathrm{supp}_k(f).
\]
\end{definition}

At a horizon event, certain places stop contributing to the defect once
the observational resolution is lowered; such events mark qualitative
changes in the comparison landscape.

\begin{proposition}[Defect drop at horizon events]
\label{prop:defect-drop}
If $k\preceq k'$ and a horizon event occurs, then
\[
  C_{k'}(f) < C_k(f)
\]
provided there exists
$p\in\mathrm{supp}_k(f)\setminus\mathrm{supp}_{k'}(f)$ with $C_{k,p}(f)>0$.
\end{proposition}

\begin{proof}
By local decomposition,
\[
  C_k(f) = \sum_{p\in\mathrm{supp}_k(f)} C_{k,p}(f),
  \qquad
  C_{k'}(f) = \sum_{p\in\mathrm{supp}_{k'}(f)} C_{k',p}(f).
\]
If a place $p$ disappears from the support and satisfies $C_{k,p}(f)>0$,
then its contribution is absent from the sum for $C_{k'}(f)$, yielding
strict inequality.
\end{proof}

\subsection{Horizon as a coarse--graining boundary}

The terminology is justified by the picture that the horizon is the
boundary beyond which certain distinctions become invisible. As
coarse--graining proceeds, the support contracts and the defect
decreases monotonically. In the extreme case where the support becomes
empty, the defect must vanish.

\begin{corollary}[Support collapse implies defect collapse]
\label{cor:support-collapse}
If $\mathrm{supp}_k(f)=\varnothing$ for some $k$, then $C_k(f)=0$.
\end{corollary}

\begin{proof}
Immediate from the local decomposition.
\end{proof}

\subsection{Interpretation}

Support and horizon give a geometric intuition for comparison:
\begin{itemize}
  \item support identifies the loci of nontrivial discrepancy;
  \item coarse--graining shrinks support and reduces defect;
  \item horizon events mark qualitative transitions in visibility;
  \item support collapse corresponds to convergence toward the baseline
        (BH) regime.
\end{itemize}

This perspective will be used in Sections~6 and~7 to analyze external
inputs and their propagation.

\section{External Input}

Sections~2--5 describe how defects behave under saturation, observation,
localization, and coarse--graining. By themselves, however, these
structural properties do not yield quantitative bounds relating defects
to height--type functionals. Such inequalities necessarily depend on
problem--specific analytic or arithmetic estimates.

This section formalizes the notion of an \emph{external input} (EI): a
collection of local bounds at a chosen base level which, once supplied,
serves as the unique problem--dependent ingredient of the theory. The
propagation results of Section~7 then transport this input throughout
the entire tower.

\subsection{Height--type functionals}

For each level $k\in K$, fix a nonnegative functional
\[
  \mathrm{Rad}_k : \mathrm{Mor}(\mathcal C) \to \mathbb R_{\ge 0},
\]
called the \emph{height} at level $k$. In applications, $\mathrm{Rad}_k$
often arises from a system of local weights $w_k(p)\ge 0$ via
\[
  \mathrm{Rad}_k(f)
    = \sum_{p\in\mathcal P} w_k(p)\,\mathbf 1_{\{p\in\mathrm{supp}_k(f)\}},
\]
although no particular form will be used here.

\subsection{Local budgets and exceptions}

Fix a base level $k_0\in K$ and a parameter $\varepsilon>0$. The aim is
to bound $C_{k_0}(f)$ in terms of $\mathrm{Rad}_{k_0}(f)$ by controlling
local contributions $C_{k_0,p}(f)$.

\begin{definition}[Local budget]
An \emph{$(\varepsilon,k_0)$--budget} is a family of inequalities
\[
  C_{k_0,p}(f) \le (1+\varepsilon)\,w_{k_0}(p)
  \qquad\text{for all but finitely many } p\in\mathcal P.
\]
Equivalently, the exceptional set
\[
  E_{\varepsilon,k_0}(f)
    := \{\, p\in\mathcal P : C_{k_0,p}(f) > (1+\varepsilon)\,w_{k_0}(p) \,\}
\]
is required to be finite.
\end{definition}

Exceptional places are those at which the defect exceeds the prescribed
budget; their contribution must be handled separately.

\subsection{Excess control}

\begin{definition}[Excess bound]
An \emph{excess bound} at $(\varepsilon,k_0)$ is a constant
$K_{\varepsilon,k_0}\ge 0$ such that, for all morphisms $f$,
\[
  \sum_{p\in E_{\varepsilon,k_0}(f)} C_{k_0,p}(f)
    \le K_{\varepsilon,k_0}.
\]
\end{definition}

Combining the budgeted and exceptional contributions yields the seed
inequality.

\begin{proposition}[Seed inequality]
\label{prop:seed-EI}
If an $(\varepsilon,k_0)$--budget and an excess bound
$K_{\varepsilon,k_0}$ are available, then for all morphisms $f$,
\[
  C_{k_0}(f)
    \le (1+\varepsilon)\,\mathrm{Rad}_{k_0}(f)
      + K_{\varepsilon,k_0}.
\]
\end{proposition}

\begin{proof}
Write
$C_{k_0}(f)=\sum_{p\notin E_{\varepsilon,k_0}(f)} C_{k_0,p}(f) +
\sum_{p\in E_{\varepsilon,k_0}(f)} C_{k_0,p}(f)$. Use the budget on the
first sum and the excess bound on the second.
\end{proof}

The inequality of Proposition~\ref{prop:seed-EI} is the \emph{external
input} required for propagation.

\subsection{Failure modes}

The EI may fail for several distinct reasons, all of which are
structural and independent of the specific application:

\begin{enumerate}
  \item[\textnormal{(F1)}] \emph{Infinite exceptions:}
  the set $E_{\varepsilon,k_0}(f)$ is infinite.

  \item[\textnormal{(F2)}] \emph{Unbounded excess:}
  the exceptional sum cannot be bounded uniformly.

  \item[\textnormal{(F3)}] \emph{Weight mismatch:}
  the chosen weights $w_{k_0}(p)$ do not reflect the scale of
  $C_{k_0,p}(f)$.

  \item[\textnormal{(F4)}] \emph{Tower incompatibility:}
  the height functionals fail to satisfy the monotonicity
  $\mathrm{Rad}_{k'}(f)\le \mathrm{Rad}_k(f)$ for $k\preceq k'$.
\end{enumerate}

These are the only obstructions to establishing the seed inequality.
Once EI holds at the base level, Section~7 guarantees its propagation
throughout the tower.

\subsection{Interpretation}

The external input isolates the analytic or arithmetic core of a
comparison problem:
\begin{itemize}
  \item the HCC structure determines the formal behavior of defects;
  \item EI supplies quantitative bounds at a single level $k_0$;
  \item once supplied, those bounds propagate automatically.
\end{itemize}

In this way, EI is the unique interface between the categorical
framework and problem--specific estimates; the remaining arguments are
purely structural.

\section{Stability and Propagation}

Once an external input (EI) is established at a single observational
level, the structural axioms of a hyper comparison category ensure that
the resulting inequality is stable under composition, localization, and
coarse--graining. This section records these stability properties and
proves the propagation theorem: a one--shot estimate at a base level
extends automatically to all higher levels in the tower.

\subsection{Stability under composition}

Defects are subadditive under composition (Theorem~\ref{thm:MCT}(4)):
\[
  C_k(g\circ f) \le C_k(f) + C_k(g).
\]
Assume likewise that the height--type functionals satisfy
\[
  \mathrm{Rad}_k(g\circ f)
    \le \mathrm{Rad}_k(f) + \mathrm{Rad}_k(g),
\]
for instance when the weights $w_k(p)$ are additive over supports.

\begin{proposition}[Compositional stability]
\label{prop:composition-stability}
Fix $\varepsilon>0$ and $K\ge 0$. If
\[
  C_k(h) \le (1+\varepsilon)\,\mathrm{Rad}_k(h) + K
\]
holds for all $h$ in a class $\mathcal G\subseteq\mathrm{Mor}(\mathcal C)$,
then the same inequality holds for any finite composite of morphisms in
$\mathcal G$.
\end{proposition}

\begin{proof}
Iterate subadditivity for $C_k$ and the corresponding subadditivity for
$\mathrm{Rad}_k$.
\end{proof}

In applications, this reduces verification of EI to a convenient
generating set.

\subsection{Stability under localization}

Suppose defects and heights admit compatible local decompositions at
level $k$.

\begin{proposition}[Local stability]
\label{prop:local-stability}
Assume the seed inequality at level $k$,
\[
  C_k(f) \le (1+\varepsilon)\,\mathrm{Rad}_k(f) + K,
\]
holds for all morphisms $f$. Then for each $p\in\mathcal P$ there exist
constants $K_{k,p}\ge 0$ with $\sum_{p\in\mathcal P} K_{k,p}\le K$ such
that
\[
  C_{k,p}(f) \le (1+\varepsilon)\,w_k(p) + K_{k,p}
\]
for all $f$.
\end{proposition}

\begin{proof}
Decompose both sides into sums over $p\in\mathcal P$ and absorb any
exceptional contribution into $K_{k,p}$.
\end{proof}

\subsection{Stability under coarse--graining}

Tower monotonicity (Theorem~\ref{thm:MCT}(5)) yields:

\begin{proposition}[Tower stability]
\label{prop:tower-stability}
If $k\preceq k'$ and the seed inequality holds at level $k$, then
\[
  C_{k'}(f) \le C_k(f)
  \qquad\text{and}\qquad
  \mathrm{Rad}_{k'}(f) \le \mathrm{Rad}_k(f)
\]
for all morphisms $f$.
\end{proposition}

Coarse--graining therefore cannot destroy an inequality; it can only
improve it.

\subsection{Propagation}

\begin{theorem}[Propagation of external input]
\label{thm:propagation}
Fix $\varepsilon>0$ and a base level $k_0\in K$. Suppose the seed
inequality
\[
  C_{k_0}(f)
    \le (1+\varepsilon)\,\mathrm{Rad}_{k_0}(f)
      + K_{\varepsilon,k_0}
\]
holds for all morphisms $f$. Then for every $k\succeq k_0$,
\[
  C_k(f)
    \le (1+\varepsilon)\,\mathrm{Rad}_k(f)
      + K_{\varepsilon,k_0}.
\]
\end{theorem}

\begin{proof}
By tower monotonicity, $C_k(f)\le C_{k_0}(f)$. By monotonicity of the
height, $\mathrm{Rad}_{k_0}(f)\ge \mathrm{Rad}_k(f)$. Substitute the seed
inequality at level $k_0$.
\end{proof}

\subsection{Propagation from generators}

\begin{corollary}
\label{cor:generator-propagation}
If the seed inequality holds at level $k_0$ for all morphisms in a class
$\mathcal G$ whose elements generate $\mathrm{Mor}(\mathcal C)$ under
finite composition, then it holds for all morphisms at every level
$k\succeq k_0$.
\end{corollary}

\begin{proof}
Combine Proposition~\ref{prop:composition-stability} with
Theorem~\ref{thm:propagation}.
\end{proof}

\subsection{Interpretation}

The tower acts as a stabilizing mechanism:
\begin{itemize}
  \item EI is required only once, at a single base level;
  \item all higher levels inherit the inequality automatically;
  \item composition and localization preserve admissible bounds;
  \item coarse--graining can only improve them.
\end{itemize}

This principle underlies the protocol of Section~8.

\section{Protocol}

The preceding sections separate comparison problems into two components:
(1) a universal structural part encoded by the HCC axioms, and (2) a
problem--dependent analytic/arithmetic part supplied as an external
input (EI). The purpose of this section is to assemble these ingredients
into a practical procedure for deriving global comparison inequalities.

\medskip

The protocol has five steps (Steps~0--4): structural verification, BH
normalization, EI at a seed level, propagation, and (when available)
convergence.

\subsection{Step 0: Structural verification}

Before applying any quantitative estimate, check that the ambient data
indeed form a hyper comparison category and that the quantitative
functionals used in EI are compatible with the tower. Concretely, one
typically verifies:

\begin{itemize}
  \item defect monotonicity under coarse--graining;
  \item local decomposition of defects and heights;
  \item subadditivity under composition;
  \item compatibility of projection towers;
  \item finiteness of support at each level.
\end{itemize}

If any of these fail, the propagation mechanism of Section~7 cannot be
invoked.

\subsection{Step 1: BH normalization}

Given $f : X \to Y$, the BH normal form (Theorem~\ref{thm:BH-normal-form})
provides the canonical factorization
\[
  X \xrightarrow{\eta_X} SX
    \xrightarrow{Sf} SY
    \xleftarrow{\eta_Y} Y.
\]
All ambiguity in comparing $X$ and $Y$ is concentrated in the gauge
$2$--cells attached to $Sf$. Accordingly, the defect $C_k(f)$ is a
structural invariant determined by the HCC data alone.

\subsection{Step 2: External input at a seed level}

Choose a base observational level $k_0\in K$. The only
problem--dependent task is to prove an EI inequality at level $k_0$:
\[
  C_{k_0}(f)
    \le (1+\varepsilon)\,\mathrm{Rad}_{k_0}(f)
      + K_{\varepsilon,k_0}.
\]
As in Section~6, this typically consists of:

\begin{itemize}
  \item a local budget for $C_{k_0,p}(f)$ outside a finite exceptional set;
  \item an excess bound controlling the total contribution of exceptional
        places.
\end{itemize}

Once these are established, the seed inequality follows
(Proposition~\ref{prop:seed-EI}).

\subsection{Step 3: Propagation}

The propagation theorem (Theorem~\ref{thm:propagation}) upgrades the seed
estimate at $k_0$ to every higher level in the tower:
\[
  C_k(f)
    \le (1+\varepsilon)\,\mathrm{Rad}_k(f)
      + K_{\varepsilon,k_0}
  \qquad\text{for all } k\succeq k_0.
\]
No further analytic input is required: tower monotonicity forces both
$C_k$ and $\mathrm{Rad}_k$ to decrease under coarse--graining.

If a generating class $\mathcal G$ is available, it suffices to verify EI
on $\mathcal G$; the resulting inequality then extends to all morphisms
(Corollary~\ref{cor:generator-propagation}).

\subsection{Step 4: Convergence}

If the tower admits eventual support collapse---i.e.\ if for each $f$ there
exists $k_\infty\succeq k_0$ with $\mathrm{supp}_{k_\infty}(f)=\varnothing$---
then Corollary~\ref{cor:support-collapse} implies
\[
  C_{k_\infty}(f)=0.
\]
Thus the comparison converges to the BH regime at sufficiently coarse
levels.

\subsection{Summary}

\begin{enumerate}
  \item Verify the HCC axioms and quantitative compatibility.
  \item Normalize via the BH factorization.
  \item Establish EI at a single seed level $k_0$.
  \item Propagate the inequality throughout the tower.
  \item When support collapses, conclude convergence to defect--zero.
\end{enumerate}

\section{Applications}

The hyper comparison category (HCC) formalism is designed to isolate the
structural mechanisms underlying comparison across multiple
observational levels. Since the axioms are minimal and purely
categorical, the framework applies broadly. The examples below
illustrate how the protocol of Section~8 organizes global comparison
problems into local estimates together with their propagation.

\subsection{General applications}

Many systems naturally come with multiple observational or
coarse--grained levels and projection maps between them; in such
situations, HCC--type data often arise essentially for free.

\paragraph{(1) Multi--resolution analysis.}
Hierarchies of function spaces (e.g.\ wavelet towers, Sobolev scales)
provide natural projection functors. Saturation corresponds to a
canonical normalization (e.g.\ orthogonal projection), while defects
measure discrepancies between representations at different resolutions.
Propagation then yields stability of bounds across scales.

\paragraph{(2) Coarse--graining in physics.}
Renormalization--group flows produce towers of effective theories.
Saturation corresponds to fixing a renormalization scheme, projection to
integrating out degrees of freedom, and defects quantify the mismatch
between coarse--grained and saturated dynamics. External inputs arise
from local energy or coupling bounds.

\paragraph{(3) Model comparison in machine learning.}
Different architectures or feature maps define observational levels.
Saturation corresponds to canonical preprocessing, projection to feature
extraction, and defects measure the noncommutativity of these
operations. Local budgets correspond to per--feature error bounds.

\paragraph{(4) Interpretation selection in cognitive systems.}
Competing interpretations of a stimulus can be modeled by different
saturation operators. Projection extracts observable behavior, and
defects measure interpretation--dependent discrepancies. Horizon events
correspond to the loss of discriminability under coarse observation.

These examples share the same pattern: comparison becomes well--behaved
only after saturation, and defects propagate predictably along towers of
observations.

\subsection{Arithmetic applications}

Arithmetic settings often come equipped with natural local decompositions
(e.g.\ primes, valuations) and height--type functionals. The HCC
framework provides a uniform language for organizing such data.

\paragraph{(1) Height inequalities.}
Local contributions at each prime define $C_{k,p}(f)$, while the height
$\mathrm{Rad}_k(f)$ aggregates prescribed weights. An external input at a
single level then yields global height bounds by propagation.

\paragraph{(2) Local--global principles.}
Local budgets control contributions at almost all primes, while excess
bounds control finitely many exceptional ones. The seed inequality then
produces global estimates.

\paragraph{(3) abc--type inequalities.}
In many contexts, the defect measures the complexity of a morphism and
the height measures the size of its support. The HCC protocol reduces
global inequalities of the form
\[
  \mathrm{defect} \le (1+\varepsilon)\,\mathrm{height} + C
\]
to verifying local bounds at a single level.

These applications indicate that HCC captures the structural essence of
local--global comparison in arithmetic.

\subsection{IUT--type comparison architectures (brief)}

Certain comparison architectures in arithmetic geometry exhibit the same
structural pattern encoded in the HCC axioms. Without assuming any
specific theory, one can view the following correspondences as a neutral
structural dictionary:

\begin{itemize}
  \item saturation models fixing an interpretation or abstraction before
        comparison;
  \item projection functors encode observable data at various levels;
  \item gauge $2$--cells record the noncommutativity between saturation
        and observation;
  \item local decomposition reflects contributions from primes (or other
        independent places);
  \item towers encode hierarchies of coarse--grained or partially
        interpreted structures.
\end{itemize}

In such settings, the HCC protocol provides a reusable template for
organizing long comparison chains: normalize via saturation, establish
local estimates at a seed level, and propagate across the tower.

\subsection{Interpretation}

Across these examples, HCC serves as a unifying language for comparison
problems with multiple observational levels. The protocol of Section~8
reduces global inequalities to local bounds, while propagation ensures
that once an estimate holds at a single level, it holds uniformly across
the hierarchy.

\section{Conclusion}

We introduced a categorical framework for comparison across multiple
observational (or coarse--grained) levels. The guiding principle is that
comparison becomes well behaved once three operations are disentangled:
\emph{saturation} (fixing an interpretation), \emph{projection}
(extracting observable data), and \emph{gauge} (recording their
noncommutativity). From these data one obtains quantitative invariants
such as defects, supports, and height--type functionals.

The HCC axioms force robust structural behavior: defects admit local
decompositions, decrease under coarse--graining, and interact
predictably with composition and localization. These mechanisms culminate
in the propagation theorem: once a suitable external input is available
at a single seed level, the corresponding inequality automatically holds
throughout the tower. The protocol of Section~8 packages this logic into
an actionable template that reduces global comparison problems to local
bounds together with their propagation.

A central feature is the separation between universal categorical
structure and problem--dependent analytic or arithmetic estimates. The
former are encoded entirely in the HCC axioms; the latter enter only via
external input at a chosen base level. This division allows the same
structural machinery to be reused across diverse settings, from
multi--resolution analysis and coarse--graining in physics to arithmetic
height inequalities and other local--global phenomena.

More broadly, HCC provides a unifying language for comparison across
scales, clarifying how local information controls global behavior once
stability and monotonicity are built into the ambient structure. We
expect further applications in contexts where interpretation,
observation, and coarse--graining interact in nontrivial ways.

\appendix
\appendix

\section*{Appendix A. Examples of Saturation (Sel)}

This appendix collects several examples of saturation operators.
The purpose is not to exhaust all possibilities, but to illustrate
how the abstract notion of an idempotent monad arises naturally in
contexts where comparison requires fixing an interpretation,
normalization, or coarse--graining before observable data can be
meaningfully extracted.

\subsection*{A.1 Reflective subcategories}

Let $\mathcal C$ be a category and $\mathcal D\subseteq\mathcal C$ a
reflective subcategory.  The reflector
\[
  S : \mathcal C \to \mathcal D \hookrightarrow \mathcal C
\]
is an idempotent monad, with unit $\eta_X : X \to SX$ exhibiting $SX$ as
the universal approximation of $X$ inside $\mathcal D$.
Typical examples include:

\begin{itemize}
  \item abelianization of groups;
  \item sheafification of presheaves;
  \item completion of metric spaces;
  \item normalization of schemes.
\end{itemize}

In each case, $SX$ represents the canonical form of $X$ with respect to a
chosen interpretation.

\subsection*{A.2 Normalization procedures}

Many constructions in geometry and analysis produce canonical
normalizations that are idempotent by design.  Examples include:

\begin{itemize}
  \item orthogonal projection onto a closed subspace of a Hilbert space;
  \item taking the reduced subscheme of a scheme;
  \item passing to the closure of a subset in a topological space;
  \item projecting a matrix onto a fixed subalgebra (e.g.\ diagonal part).
\end{itemize}

These operations enforce a structural constraint before comparison.

\subsection*{A.3 Coarse--graining and abstraction}

In settings with multiple levels of description, saturation may represent
the act of fixing an interpretation or abstraction.  Examples include:

\begin{itemize}
  \item selecting a coordinate system or gauge in physics;
  \item canonical preprocessing of data (e.g.\ centering, whitening);
  \item abstraction maps in logic or type theory;
  \item quotienting by symmetries or equivalence relations.
\end{itemize}

Here $SX$ is the canonical representative of $X$ under the chosen
abstraction.

\subsection*{A.4 Algebraic and arithmetic examples}

In arithmetic geometry, saturation often corresponds to imposing a
canonical structure before comparison.  Examples include:

\begin{itemize}
  \item passing from a ring to its integral closure;
  \item taking the Néron model of an abelian variety;
  \item forming the maximal unramified extension of a local field;
  \item normalizing divisors or line bundles.
\end{itemize}

These operations ensure that subsequent comparisons reflect intrinsic
rather than incidental features.

\subsection*{A.5 Information--theoretic examples}

In information processing, saturation corresponds to canonical
preprocessing steps that remove irrelevant variation:

\begin{itemize}
  \item projecting data onto a feature subspace;
  \item enforcing invariances (e.g.\ translation or scale invariance);
  \item canonical encoding of signals;
  \item normalization of probability distributions.
\end{itemize}

Such operations are idempotent and serve as interpretation--fixing
preparations for comparison.

\medskip

Across all these examples, saturation plays the same conceptual role:
it produces a canonical form of an object, eliminating interpretational
ambiguity and enabling meaningful comparison at the observable level.

\section*{Appendix B. Examples of Projection Towers}

This appendix provides examples of projection functors and tower
structures.  The purpose is to illustrate how hierarchies of
observational levels arise naturally in many mathematical and scientific
contexts, and how coarse--graining manifests as a functorial passage to
coarser categories.

\subsection*{B.1 Multi--resolution analysis}

Let $\mathcal C$ be a space of functions (e.g.\ $L^2(\mathbb R)$) and let
$\mathcal O_k$ denote the subspace spanned by wavelets at scales
$\ge k$.  The projection
\[
  \Pi_k : \mathcal C \to \mathcal O_k
\]
is the orthogonal projection onto the coarse resolution $k$.  The tower
maps
\[
  \sigma_{k\to k'} : \mathcal O_k \to \mathcal O_{k'}
  \qquad (k\preceq k')
\]
are the natural inclusions of coarser scales.  Coarse--graining removes
fine detail, and defects measure the mismatch between normalized and
observed representations.

\subsection*{B.2 Coarse--graining in physics}

In renormalization group (RG) theory, one considers a hierarchy of
effective theories obtained by integrating out high--energy degrees of
freedom.  Let $\mathcal C$ be the category of microscopic models, and
$\mathcal O_k$ the category of effective theories at scale $k$.  The
projection
\[
  \Pi_k : \mathcal C \to \mathcal O_k
\]
is the RG flow map, and the tower maps
\[
  \sigma_{k\to k'} : \mathcal O_k \to \mathcal O_{k'}
\]
represent further coarse--graining.  The HCC axioms reflect the physical
principle that coarse--graining suppresses fine distinctions.

\subsection*{B.3 Feature hierarchies in machine learning}

Let $\mathcal C$ be a space of data objects (e.g.\ images).  A feature
extractor at level $k$ produces a representation
\[
  \Pi_k : \mathcal C \to \mathcal O_k,
\]
where $\mathcal O_k$ is the feature space at depth $k$ of a neural
network.  Deeper layers correspond to coarser, more abstract features.
The tower maps
\[
  \sigma_{k\to k'} : \mathcal O_k \to \mathcal O_{k'}
\]
are induced by the network architecture.  Defects measure the
noncommutativity between preprocessing (saturation) and feature
extraction.

\subsection*{B.4 Valuation towers}

Let $\mathcal C$ be a category of arithmetic objects (e.g.\ number
fields, schemes).  For each $k$, let $\mathcal O_k$ encode data visible
at a set of valuations of bounded complexity (e.g.\ primes of norm
$\le k$).  The projection
\[
  \Pi_k : \mathcal C \to \mathcal O_k
\]
forgets contributions from valuations of complexity $>k$.  The tower maps
\[
  \sigma_{k\to k'} : \mathcal O_k \to \mathcal O_{k'}
\]
add back valuations of intermediate complexity.  This produces a natural
local--global hierarchy.

\subsection*{B.5 Abstract poset--indexed towers}

More generally, let $(K,\preceq)$ be any poset and let
$\{\mathcal O_k\}_{k\in K}$ be a diagram of categories with functors
\[
  \sigma_{k\to k'} : \mathcal O_k \to \mathcal O_{k'}
  \qquad (k\preceq k')
\]
satisfying the coherence conditions of a functor from $K$ to
$\mathbf{Cat}$.  Any family of projections
\[
  \Pi_k : \mathcal C \to \mathcal O_k
\]
compatible with the tower maps defines a projection tower in the sense of
the HCC axioms.

\medskip

These examples demonstrate that projection towers arise naturally in
contexts where information is organized hierarchically.  The HCC
framework abstracts the common structural features of such hierarchies
and provides a unified language for comparison across observational
levels.

\section*{Appendix C. Defect, Support, and Height: Additional Examples}

This appendix provides concrete illustrations of the quantitative
invariants introduced in Sections~3--6.  The goal is to clarify how
defects arise from gauge $2$--cells, how support identifies the loci of
nontrivial discrepancy, and how height--type functionals aggregate local
information.

\subsection*{C.1 A simple defect computation}

Let $\mathcal O_k$ be a category enriched in groupoids, and let
$\alpha : A \Rightarrow B$ be an invertible $2$--cell.  Suppose the size
functional $|\cdot|_k$ is given by
\[
  |\alpha|_k = \ell(\alpha),
\]
where $\ell$ is a length function on the automorphism group of $A$.
Then for a morphism $f : X \to Y$ in $\mathcal C$, the defect
\[
  C_k(f) = |\Phi_f^{(k)}|_k
\]
measures the minimal ``cost'' of adjusting the comparison diagram so that
saturation and observation commute.

Even in this simple setting, defects encode the essential obstruction to
direct comparison.

\subsection*{C.2 Support: finite and infinite cases}

Let $\mathcal P$ be a set of places (e.g.\ primes, features, channels).
For a morphism $f$, the localized defects
\[
  C_{k,p}(f) = |\mathrm{Loc}_{k,p}(\Phi_f^{(k)})|_{k,p}
\]
identify the contributions from each place.

\paragraph{Finite support.}
If $C_{k,p}(f)=0$ for all but finitely many $p$, then
\[
  \mathrm{supp}_k(f)
    = \{\, p : C_{k,p}(f)>0 \,\}
\]
is finite.  This is typical in arithmetic settings where only finitely
many primes contribute to a given comparison.

\paragraph{Infinite support.}
In analytic or geometric settings, support may be infinite.  For
instance, if $C_{k,p}(f)$ decays rapidly with $p$, then
\[
  C_k(f) = \sum_{p\in\mathcal P} C_{k,p}(f)
\]
converges even though $\mathrm{supp}_k(f)$ is infinite.  The HCC axioms
accommodate both behaviors.

\subsection*{C.3 Horizon events}

Consider a tower $k\preceq k'\preceq k''$.  Suppose
\[
  \mathrm{supp}_k(f) = \{p_1,p_2,p_3\},\qquad
  \mathrm{supp}_{k'}(f) = \{p_1,p_2\},\qquad
  \mathrm{supp}_{k''}(f) = \{p_1\}.
\]
Then the transitions
\[
  \mathrm{supp}_k(f)\supsetneq\mathrm{supp}_{k'}(f)
  \quad\text{and}\quad
  \mathrm{supp}_{k'}(f)\supsetneq\mathrm{supp}_{k''}(f)
\]
are horizon events.  At each horizon, the defect drops by the
contribution of the disappearing places:
\[
  C_{k'}(f) = C_k(f) - C_{k,p_3}(f),\qquad
  C_{k''}(f) = C_{k'}(f) - C_{k',p_2}(f).
\]
This illustrates how coarse--graining suppresses fine distinctions.

\subsection*{C.4 Height functionals}

Height--type functionals aggregate local weights.  A typical example is
\[
  \mathrm{Rad}_k(f)
    = \sum_{p\in\mathrm{supp}_k(f)} w_k(p),
\]
where $w_k(p)\ge 0$ is a prescribed weight.

\paragraph{Uniform weights.}
If $w_k(p)=1$ for all $p$, then $\mathrm{Rad}_k(f)$ counts the size of
the support.

\paragraph{Complexity weights.}
If $w_k(p)$ increases with the complexity of $p$, then
$\mathrm{Rad}_k(f)$ measures the ``weighted size'' of the support.

\paragraph{Decay weights.}
If $w_k(p)$ decreases with $p$, then $\mathrm{Rad}_k(f)$ emphasizes
low--complexity contributions.

The HCC framework does not impose any specific choice of weights; only
monotonicity under coarse--graining is required for propagation.

\subsection*{C.5 Local decomposition in practice}

Let $f : X \to Y$ be a morphism with defect
\[
  C_k(f) = 7.3.
\]
Suppose the localized contributions are
\[
  C_{k,p_1}(f)=3.1,\quad
  C_{k,p_2}(f)=2.0,\quad
  C_{k,p_3}(f)=2.2,
\]
and $C_{k,p}(f)=0$ for all other $p$.  Then
\[
  \mathrm{supp}_k(f)=\{p_1,p_2,p_3\},
\]
and the height
\[
  \mathrm{Rad}_k(f)
    = w_k(p_1)+w_k(p_2)+w_k(p_3)
\]
aggregates the weights of these places.

This decomposition is the basis for the external input: local budgets
control each $C_{k,p}(f)$, and excess bounds control the exceptional
places.

\medskip

These examples illustrate how defects, supports, and height--type
functionals behave in concrete settings.  They provide the intuition
behind the structural results of Sections~5--7 and the protocol of
Section~8.

\section*{Appendix D. Categorical Background}

This appendix summarizes the categorical notions used in the definition
of a hyper comparison category.  The material is standard, and the
presentation is tailored to the needs of the main text; see
\cite{MacLane1998,Leinster2004,Street1972,Borceux1994} for background.

\subsection*{D.1 Idempotent monads and reflective subcategories}

A monad on a category $\mathcal C$ consists of an endofunctor
$S : \mathcal C \to \mathcal C$ together with natural transformations
\[
  \eta : \mathrm{Id}_{\mathcal C} \Rightarrow S,
  \qquad
  \mu : S^2 \Rightarrow S,
\]
satisfying the usual associativity and unit axioms.  The monad is
\emph{idempotent} if $\mu$ is an isomorphism.

Idempotent monads correspond precisely to reflective subcategories:
if $\mathcal D\subseteq\mathcal C$ is reflective with reflector
$S : \mathcal C \to \mathcal D$, then $S$ extends to an idempotent monad
on $\mathcal C$.  Conversely, the fixed points of an idempotent monad
form a reflective subcategory.

In the HCC framework, saturation is modeled by an idempotent monad.

\subsection*{D.2 2-categories and invertible 2-cells}

A 2-category consists of:

\begin{itemize}
  \item objects;
  \item 1-morphisms between objects;
  \item 2-morphisms between 1-morphisms.
\end{itemize}

Composition of 1-morphisms is associative up to coherent 2-isomorphism,
and 2-morphisms compose both vertically and horizontally.

In a hyper comparison category, the observable categories $\mathcal O_k$
are treated as 2-categories (or categories enriched in groupoids), and
the gauge $\Phi_f^{(k)}$ is an invertible 2-cell measuring the failure of
saturation and observation to commute.

\subsection*{D.3 Localization functors}

Given a category $\mathcal O_k$ and a set of ``places'' $\mathcal P$, a
localization functor
\[
  \mathrm{Loc}_{k,p} : \mathcal O_k \to \mathcal O_{k,p}
\]
extracts the contribution of the place $p$.  These functors are required
to be compatible with the size functionals:
\[
  |\alpha|_k = \sum_{p\in\mathcal P} |\mathrm{Loc}_{k,p}(\alpha)|_{k,p}.
\]

Localization is used to define support and to decompose defects into
local contributions.

\subsection*{D.4 Size functionals on 2-cells}

For each level $k$, the size functional
\[
  |\cdot|_k : \{\text{invertible 2-cells in }\mathcal O_k\}
    \to \mathbb R_{\ge 0}
\]
assigns a nonnegative real number to each invertible 2-cell.  The
functional is required to satisfy:

\begin{itemize}
  \item invariance under isomorphism of 2-cells;
  \item subadditivity under vertical and horizontal composition;
  \item compatibility with localization;
  \item monotonicity under coarse--graining.
\end{itemize}

These conditions ensure that defects behave predictably under the
operations of the HCC.

\subsection*{D.5 Poset-indexed diagrams and coherence}

Let $(K,\preceq)$ be a poset.  A $K$-indexed diagram of categories
consists of:

\begin{itemize}
  \item a category $\mathcal O_k$ for each $k\in K$;
  \item functors $\sigma_{k\to k'} : \mathcal O_k \to \mathcal O_{k'}$
        for each $k\preceq k'$;
  \item coherence conditions
        \[
          \sigma_{k\to k} = \mathrm{Id},
          \qquad
          \sigma_{k\to k''}
            = \sigma_{k'\to k''}\circ\sigma_{k\to k'}
            \quad (k\preceq k'\preceq k'').
        \]
\end{itemize}

Such diagrams model hierarchies of observational levels.  The projection
functors $\Pi_k : \mathcal C \to \mathcal O_k$ are required to satisfy
\[
  \Pi_{k'} = \sigma_{k\to k'} \circ \Pi_k,
\]
ensuring that observation commutes with coarse--graining.

\medskip

These categorical notions provide the structural foundation for the HCC
framework.  They ensure that saturation, observation, and gauge interact
coherently, and that defects behave predictably under localization,
composition, and coarse--graining.

\section*{Appendix E. Structural Correspondence with IUT-type Architectures (Optional)}

This appendix outlines a purely structural correspondence between the
components of a hyper comparison category (HCC) and certain comparison
architectures that arise in arithmetic geometry.  The purpose is not to
assert any mathematical claims about specific theories, but simply to
illustrate how the abstract mechanisms of saturation, projection, gauge,
local decomposition, and propagation may be recognized in settings with
similar structural features.

The correspondence is schematic and depends only on the formal pattern
of comparison across multiple observational levels.

\subsection*{E.1 Saturation and interpretation fixing}

In HCC, saturation is an idempotent monad
\[
  S : \mathcal C \to \mathcal C
\]
that fixes an interpretation before comparison.  In IUT-type
architectures, one encounters analogous procedures in which objects are
first placed into a canonical or interpretation-stable form before
observable data are extracted.  The role of $SX$ is played by the
interpretation-fixed version of $X$.

\subsection*{E.2 Projection and observable data}

Projection functors
\[
  \Pi_k : \mathcal C \to \mathcal O_k
\]
extract observable shadows at level $k$.  In IUT-type settings, various
“visible” invariants or partially interpreted structures serve a similar
role, providing the data on which comparisons are performed.

\subsection*{E.3 Gauge and comparison discrepancy}

The gauge $2$-cells
\[
  \Phi_f^{(k)} :
    \Pi_k(Sf)\circ\Pi_k(\eta_X)
    \Rightarrow
    \Pi_k(\eta_Y)\circ\Pi_k(f)
\]
measure the failure of saturation and observation to commute.  In
IUT-type architectures, comparison often involves reconciling data
obtained after different interpretation or abstraction steps.  The
resulting discrepancies play the same structural role as the gauge.

\subsection*{E.4 Local decomposition}

In HCC, defects decompose over places:
\[
  C_k(f) = \sum_{p\in\mathcal P} C_{k,p}(f).
\]
IUT-type settings frequently involve contributions indexed by primes,
valuations, or other local data.  The structural similarity lies in the
existence of independent local channels whose contributions aggregate to
a global invariant.

\subsection*{E.5 Towers and hierarchical comparison}

The projection tower
\[
  \Pi_{k'} = \sigma_{k\to k'}\circ\Pi_k
\]
models comparison across multiple observational levels.  IUT-type
architectures also organize comparison across hierarchies of partially
interpreted structures.  The structural pattern—comparison at one level
feeding into comparison at another—is formally similar.

\subsection*{E.6 External input}

In HCC, the only analytic or arithmetic input enters through a seed
inequality at a single level:
\[
  C_{k_0}(f)
    \le (1+\varepsilon)\,\mathrm{Rad}_{k_0}(f) + K.
\]
IUT-type settings likewise require external estimates at specific stages
of a comparison chain.  The structural role of these estimates matches
the role of the EI in the HCC protocol.

\subsection*{E.7 Propagation along the tower}

The propagation theorem shows that once the seed inequality holds at
$k_0$, it holds for all $k\succeq k_0$.  In IUT-type architectures,
comparison results often propagate along a hierarchy of structures in a
similar manner.  The HCC framework abstracts this propagation mechanism.

\medskip

This appendix is intended only as a structural guide.  The HCC formalism
is independent of any specific arithmetic theory, and the correspondences
above are schematic analogies rather than mathematical identifications.

\begin{thebibliography}{99}

\bibitem{MacLane1998}
S.\ Mac~Lane,
\emph{Categories for the Working Mathematician},
2nd ed., Graduate Texts in Mathematics 5, Springer, 1998.

\bibitem{Borceux1994}
F.\ Borceux,
\emph{Handbook of Categorical Algebra 1: Basic Category Theory},
Encyclopedia of Mathematics and its Applications 50, Cambridge University Press, 1994.

\bibitem{Street1972}
R.\ Street,
\emph{The formal theory of monads},
Journal of Pure and Applied Algebra \textbf{2} (1972), 149--168.

\bibitem{Leinster2004}
T.\ Leinster,
\emph{Higher Operads, Higher Categories},
London Mathematical Society Lecture Note Series 298, Cambridge University Press, 2004.

\bibitem{Kelly1982}
G.\ M.\ Kelly,
\emph{Basic Concepts of Enriched Category Theory},
London Mathematical Society Lecture Note Series 64, Cambridge University Press, 1982.

\bibitem{Lack2010}
S.\ Lack,
\emph{A $2$-categories companion},
in \emph{Towards Higher Categories}, IMA Volumes in Mathematics and its Applications 152,
Springer, 2010, 105--191.

\bibitem{BaezStay2011}
J.\ C.\ Baez and M.\ Stay,
\emph{Physics, topology, logic and computation: a Rosetta Stone},
in \emph{New Structures for Physics}, Lecture Notes in Physics 813,
Springer, 2011, 95--172.

\bibitem{FongSpivak2019}
B.\ Fong and D.\ I.\ Spivak,
\emph{Seven Sketches in Compositionality: An Invitation to Applied Category Theory},
Cambridge University Press, 2019.

\end{thebibliography}

\end{document}